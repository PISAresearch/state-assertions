j\PassOptionsToPackage{usenames}{xcolor}
\PassOptionsToPackage{dvipsnames}{xcolor}
\documentclass{llncs}
\usepackage[utf8]{inputenc}
\usepackage{booktabs} % For formal tables
\usepackage{multirow}
\usepackage{amssymb}
\usepackage{graphicx}
\usepackage{url}
\usepackage{xspace}
\usepackage{pifont}% http://ctan.org/pkg/pifont
\usepackage{color}
\usepackage{boxedminipage}
\usepackage[ff,sets,keys,primitives,operators]{cryptocode}
\usepackage{framed}
\usepackage[group-separator={,}]{siunitx}
\newcommand\bmmax{2}
\usepackage{bm}
\usepackage{caption}
\usepackage{hyperref}
\usepackage{footnote}
\usepackage{units}
\usepackage{multicol,lipsum}
\colorlet{iomsg}{MidnightBlue}
\colorlet{party}{brown}
\colorlet{entry}{NavyBlue}
\colorlet{string}{BlueViolet}

\newcommand{\cmark}{\ding{51}}%
\newcommand{\xmark}{\ding{55}}%

\newcommand{\instantiated}{\mathsf{instantiated}}
\newcommand{\instantiatedno}{\mathsf{NO}}
\newcommand{\instantiatedyes}{\mathsf{YES}}

\newcommand{\gamestatus}{\mathsf{phase}}
\newcommand{\gameregister}{\mathsf{INIT}}
\newcommand{\gamesetup}{\mathsf{SETUP}}
\newcommand{\gameattack}{\mathsf{ATTACK}}
\newcommand{\gamereveal}{\mathsf{REVEAL}}
\newcommand{\gamewinner}{\mathsf{WIN}}
\newcommand{\gamefraud}{\mathsf{FRAUD}}
\newcommand{\gamefinished}{\mathsf{GAMEOVER}}

\newcommand{\chanstatus}{\mathsf{status}}

\newcommand{\chanon}{\mathsf{ON}}
\newcommand{\chandispute}{\mathsf{DISPUTE}}
\newcommand{\chanoff}{\mathsf{OFF}}

\newcommand{\hready}{\mathsf{hready}}
\newcommand{\hboard}{\mathsf{hboard}}
\newcommand{\hcell}{\mathsf{hcell}}
\newcommand{\hship}{\mathsf{hship}}
\newcommand{\hshiplocation}{\mathsf{hshiplocation}}

%\newcommand{\hash}{\textsf{H}}
\newcommand{\cmd}{\mathsf{cmd}}
\newcommand{\hstate}{\mathsf{hstate}}
\newcommand{\hstatei}{\mathsf{hstate}_{\monotoniccounter}}
\newcommand{\hstateplus}{\ensuremath{\mathsf{hstate}_{\monotoniccounter+1}}}
\newcommand{\hstateminus}{\ensuremath{\mathsf{hstate}_{\monotoniccounter-1}}}
\newcommand{\monotoniccounter}{\mathsf{i}}
\newcommand{\stateinfo}{\mathsf{state}}
\newcommand{\stateinfoi}{\mathsf{state}_{\mathsf{i}}}
\newcommand{\stateinfominus}{\mathsf{state}_{\mathsf{i-1}}}
\newcommand{\stateinfoplus}{\mathsf{state}_{\mathsf{i+1}}}
\newcommand{\participant}{\mathcal{P}}

\newcommand{\rani}{\mathsf{r}_{\mathsf{i}}}
\newcommand{\ran}{\mathsf{r}}
\newcommand{\ranminus}{\mathsf{r}_{\mathsf{i-1}}}
\newcommand{\ranplus}{\mathsf{r}_{\mathsf{i+1}}}

\newcommand{\statechannel}{\mathsf{SC}}
\newcommand{\statechanneldispute}{\mathsf{SC}.\mathsf{trigger}}
\newcommand{\statechannelsetstate}{\mathsf{SC}.\mathsf{setstatehash}}
\newcommand{\statechannelresolve}{\mathsf{SC}.\mathsf{resolve}} 
\newcommand{\statechannelgetcommitment}{\mathsf{SC}.\mathsf{getstatehash}} 
\newcommand{\statechannelgetdispute}{\mathsf{SC}.\mathsf{getdispute}} 
\newcommand{\statechannelclose}{\mathsf{SC}.\mathsf{close}} 

\newcommand{\sign}{\mathsf{Sign}}
\newcommand{\verifysig}{\mathsf{VerifySig}}

\newcommand{\battleship}{\mathsf{BS}}
\newcommand{\battleshipfraud}{\mathsf{BS.fraud}}
\newcommand{\battleshipattackcell}{\mathsf{BS.attackcell}}
\newcommand{\battleshipbegin}{\mathsf{BS.begingame}}
\newcommand{\battleshipquit}{\mathsf{BS.quitgame}}
\newcommand{\battleshipcommit}{\mathsf{BS.commit}}
\newcommand{\battleshipplacebet}{\mathsf{BS.placebet}}
\newcommand{\battleshipselectboard}{\mathsf{BS.select}}
\newcommand{\battleshiprevealcell}{\mathsf{BS.opencell}}
\newcommand{\battleshipsinking}{\mathsf{BS.sunk}}
\newcommand{\battleshiprevealships}{\mathsf{BS.openships}}
\newcommand{\battleshiprevealboard}{\mathsf{BS.openships}}
\newcommand{\battleshipgameover}{\mathsf{BS.gameover}}
\newcommand{\battleshipdeposit}{\mathsf{BS.deposit}}
\newcommand{\battleshipwithdraw}{\mathsf{BS.withdraw}}
\newcommand{\battleshipfinish}{\mathsf{BS.finish}}

%\newcommand{\battleshipshipnotplaced}{\mathsf{BS.fraudnothit}}
\newcommand{\battleshipdeclarednotsunk}{\mathsf{BS.declarednotsunk}}
\newcommand{\battleshipdeclarednothit}{\mathsf{BS.declarednothit}}
\newcommand{\battleshipsamecell}{\mathsf{BS.attacksamecell}}
\newcommand{\battleshiptwoships}{\mathsf{BS.celltwoships}}
\newcommand{\battleshipchallengeexpired}{\mathsf{BS.expiredchallenge}}

\newcommand{\battleshiplock}{\mathsf{BS.lock}}
\newcommand{\battleshipunlock}{\mathsf{BS.unlock}}
\newcommand{\battleshipgetstate}{\mathsf{BS.getstate}}

\newcommand{\appcontract}{\mathsf{AC}}
\newcommand{\applock}{\mathsf{AC.lock}}
\newcommand{\appunlock}{\mathsf{AC.unlock}}

\newcommand{\timerchallenge}{\mathsf{\Delta}_{\mathsf{challenge}}}
\newcommand{\timechallenge}{\mathsf{t}_{\mathsf{challenge}}}
\newcommand{\timerextra}{\mathsf{\Delta}_{\mathsf{extra}}}
\newcommand{\timerdispute}{\mathsf{\Delta}_{\mathsf{dispute}}}
\newcommand{\timenow}{\mathsf{t}_{\mathsf{now}}}
\newcommand{\timestart}{\mathsf{t}_{\mathsf{start}}}
\newcommand{\timeend}{\mathsf{t}_{\mathsf{end}}}
\newcommand{\timedispute}{\timenow + \mathsf{\Delta}_{\mathsf{dispute}}}

% Colorful diagrams 
\newcommand{\constructor}{\textcolor{entry}{\bf constructor }}
\newcommand{\oninput}{\textcolor{entry}{\bf function }}
\newcommand{\stringlitt}[1]{\texttt{\textcolor{string}{#1}}}

\begin{document}

	\title{Tutorial 1: Principles of Security Management}
	
	\author{Patrick McCorry}
	
	\institute{Kings College London, UK\\
		\email{patrick.mccorry@kcl.ac.uk}}
	
	
	\maketitle
	\begin{abstract}
	Exercises to understand the principles of security management. 
	It is recommended to make notes of discussions on this tutorial sheet for future use. 
	\end{abstract} 

\section{Group of Appreciation of the Natterjack Toad (GANT)}

The Group of Appreciation of the Natterjack Toad (GANT) is a conservation group that is keen to promote and preserve the well-being of the Natterjack Toad. 
It is a UK-registered charity and has a significant number of members world-wide who are all keen to promote the work of GANT. 
%One fact about the natterjack toad, it is claimed to be Europe's noisiest amphibian with the male call being audiable over several kilometres. 
Unfortunately it is an endangered species that is gradulaly being destoryed by the development of new areas. 
For example, it was locally extinct in some areas of Wales due to development work and it had to be re-introduced.\footnote{https://www.denbighshirecountryside.org.uk/natterjack-toad/}

All information for the group can be accessed using a web-based application or by contacting the group's honorary secretary Dr Jane Peabody for the paper-based records. 
This information includes the group's member records, its activities, meeting places, natterjack toad habitats, confidential aspects about their work, etc. 
In the past, members have raised concerns about information assurance as the website has been previously compromised oweing to the server containing no significant security controls. 

The chairman Ms Rachel Jackson has heard about information security and believes it is the right time to take it more seriously, but she doesn't know that much about it. 
This is where you come in. Ms Rachel Jackson has hired your group to learn more about protecting their information. 

\subsection{What is Information Security?} 

Let's get the ball rolling. 
You are preparing for a meeting with Ms Rachel Jackson to convince her that information security should be taken seriously by GANT.
In preparation for the meeting, your group should agree on the following principles and points: 

\begin{itemize}
	\item What is Information Security?
	\item \textcolor{red}{The practice of preventing unauthorised access, use, disclosure, disruption, modification, inspection, recording or destruction of information}
	\item What is the focus of Information Security? 
	\item \textcolor{red}{Confidentiality, Integrity and Availability}
	\item What information assets under the control of GANT may require protection? 
	\item \textcolor{red}{Everything. This includes members records (i.e. date of birth, location, passwords), the toad's camp location (if leaked, people may attack them). etc}
\end{itemize}

It is also important to prepare a brief discussion about \textit{confidentiality vs availability} of information.
While they are conceptually opposing goals, you will need to convince Ms Rachel Jackson how there can be acceptable trade-offs for the two goals. 
Please use the informtion assets that you identified as part of the discussion. 

\textcolor{red}{Confidentiality tries to keep information secret to only those who need to know, whereas availability seeks to make information accessible.  While information can be kept secret by removing all access to it, this raises the question whether it is then useful to keep at all. There must be a tradeoff to let only parties who need access to the information. Cryptography can be used encrypt the information so it can be accessible, but only those with the decryption key can open it. }

\section{What risks may GANT face in the future?} 

The meeting with Ms Rachel Jackson has begun. 
She clearly understands the principles of Information Security, but she does not yet know how to assess threats, vulnerabilities or risk. 
To help her understand, identify three threats, vulnerabilities and risks that GANT's information assurance system needs to manage. 
Please remember that Ms Rachel Jackson is not tech savvy per say.
For example, she will not understand how a MYSQL injection attack works.
Try to use simple english to explain the above three points. 

\textcolor{red}{A threat highlights a problem that if it arises will result in adverse consequences. This may include 1) Information about members might be accessible by unauthorised people. 2) Information about the habitats of the Natterjack toad might be used by those who are not inclined to support its ongoing existence. 3) The website might be comrpomised and unofficial messages added to it. }

\textcolor{red}{A vulnerability is a weakness in the system that might allow the threat to happen.  This may include 1) Records of members are stored in an unreliable computer system that may crash in the future. 2) Information about the toad's habitats may be stored on an old internet-based and insecure server. 3) The administrator password for the website has never been changed from the default password. }

\textcolor{red}{A risk is the consequence (and likelihood) of what may happen if the vulnerability is exploited. This may include 1) Unscrupulous property developers may gain access to personal details about members and later harass them. 2) A habitat of the Natterjack toad might be destroyed by someone who is not interested in its existence. 3) Someone might gain access to the GANT website and update it with offensive information.}


\section{Financially reasonable to pursue? } 

You have helped increase Ms Rachel Jackson's awareness of dangers that may occur if there are no appropriate security controls in place.
During the discussion, she mentions that there may be times when implementing a security control is more expensive than the information asset that it is protecting. 
She is correct with this comment. 
Can you provide two examples of security controls that can be implemented for an information asset, but it may be cost-effective to do. 

\textcolor{red}{Facial recognition to physical safe. Dr Jane Peabody keeps all records on paper and it would be nice if only her face could unlock a safe to access it. However, it is more cost-effective to use biometrics or just a physical key that she keeps.}

\textcolor{red}{Locking servers in a dedicated room. It is possible to keep the servers in a dedicated room and only Dr Jane Peabody has the key to access it. This is useful to prevent side-channel attacks etc but requires a room to be dedicated for this purpose. }

\section{Security culture and professionalism} 
While Ms Rachel Jackson is ready to prepare a budgetto implement an information security management system, she wants to ensure that all employees within the organisation takes it seriously. 
Again, she asks you for advice. 
What can she do to help promote security awareness within the organisation? And what are the likely benefits of a good security culture if it works? 

\textcolor{Red}{\textbf{Training: } All employees should attend training (and hopefully fun) days about how to improve information assurance during their job. If they are properly informed about how to do it, then it should become part of the culture. }

\textcolor{Red}{\textbf{Senior Management Support: } Board members and all management should issue statements to all workers (and possibly publicly) about their commitment to information assurance. If the organisation is taking it seriously, then workers will understand its importance. }

\textcolor{Red}{\textbf{Security Champions: } Employees should be appointed as Security Champaigns across the organisation to be responsible for checking if policies are followed and reporting back any information assurance issues that arise. }

\textcolor{red}{The primary benefit of a good security culture is to reduce vulnerabilities being exploited by attacks. 
As well, it should help identify information assurance issues that arise as employees will feel their opinion matters and that they feel comfortable reporting it to senior management without fear of recourse. 
For example, if an employee's action led to an increased risk of a vulnerability, then it is best to report this early, rectify the issue and figure out how to prevent it happening again in the future.} 
\end{document}
